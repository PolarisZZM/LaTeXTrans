
\begin{figure}[t!]
    \centering
    \captionsetup{type=figure}
    \begin{subfigure}[t]{0.48\linewidth}
        \input{graphs/early_late/loss_vs_flops}
    \end{subfigure}
    \hfill
    \begin{subfigure}[t]{0.48\linewidth}
        \input{graphs/early_late/d_n_ratio_vs_flops}
    \end{subfigure}
    \vspace{-3mm}
    \caption{\textbf{原生多模态模型的缩放属性。}基于在\cref{sec:scaling_laws_early}中对缩放定律的研究,我们观察到:(1) 当使用相同的计算预算$C$(以FLOP为单位)进行训练时,早期融合模型和晚期融合模型提供相当的验证损失$L$;(2) 这种性能是通过在参数$N$和训练词元数量$D$之间进行不同的权衡来实现的,其中早期融合模型需要的参数更少。
\edit{;(3) 对于给定的FLOP预算,稀疏早期融合模型实现了更低的损失并且需要更多的训练词元。} 
    }
    \label{fig:teaser}
\end{figure}
