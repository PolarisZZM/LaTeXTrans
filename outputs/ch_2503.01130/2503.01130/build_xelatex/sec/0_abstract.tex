\begin{abstract}
房间重识别(ReID)是一项具有挑战性但至关重要的任务,广泛应用于增强现实(AR)和家庭护理机器人等领域。现有的视觉位置识别(VPR)方法通常依赖于全局描述符或聚合局部特征,但在拥挤的室内环境中,尤其是那些充满人造物体的环境中,常常难以有效工作。这些方法往往忽视了物体相关信息的关键作用。为了解决这一问题,我们提出了AirRoom,这是一种物体感知管道,整合了多层次的物体相关信息——从全局上下文到物体补丁、物体分割和关键点——并采用粗到细的检索方法。在四个新构建的数据集(MPReID、HMReID、GibsonReID和ReplicaReID)上的大量实验表明,AirRoom在几乎所有评估指标上都超越了最先进的(SOTA)模型,提升幅度从6\%到80\%不等。此外,AirRoom展现出显著的灵活性,允许管道中的各种模块被不同的替代方案替换,而不会影响整体性能。它还在不同视角变化下表现出强大且稳定的性能。项目网站:\href{https://sairlab.org/airroom/}{\textcolor[rgb]{0.3,0.5,1}{https://sairlab.org/airroom/}}。

\end{abstract}
