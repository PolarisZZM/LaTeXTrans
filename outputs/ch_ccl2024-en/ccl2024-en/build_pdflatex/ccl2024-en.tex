
\documentclass[11pt]{article}
\usepackage{CJKutf8}
\usepackage[utf8]{inputenc}

\usepackage[hyperref]{ccl2024-en}
\usepackage{times}
\usepackage{url}
\usepackage{latexsym}
\usepackage{fancyhdr}

\pagestyle{fancy}
\fancyhf{}
\lhead{Computational Linguistics}
\renewcommand{\headrulewidth}{0pt}

\title{第二十三届全国计算语言学学术会议(CCL-2024)论文集撰写指南}

\author{First Author \\
  Affiliation / Address line 1 \\
  Affiliation / Address line 2 \\
  Affiliation / Address line 3 \\
  {\tt email@domain} \\\And
  Second Author \\
  Affiliation / Address line 1 \\
  Affiliation / Address line 2 \\
  Affiliation / Address line 3 \\
  {\tt email@domain} \\}

\date{}
\begin{document}
\begin{CJK*}{UTF8}{gbsn}
\maketitle
\begin{abstract}
  本文包含了为提交至CCL-2024或已被接受刊登于其会议录中的论文准备的说明。本文本身符合其自身的规范,因此是您的手稿应呈现的样例。这些说明适用于提交审稿的论文以及接受后的最终版本。作者需遵守本文中报告的所有要求。
\end{abstract}
\section{简介}
\label{intro}

\cclfootnote{
    \hspace{-0.65cm}
    \textcopyright 2024 中国计算语言学会议

    \noindent 以创意共享署名4.0国际许可证发布
}

以下说明针对提交到CCL-2024或已被接受刊登在其会议论文集中的论文的作者。所有作者都必须遵守这些规范。作者需提供其论文的可移植文档格式(PDF)版本。 \textbf{会议论文集设计为在A4纸上打印。}

我们可能会在 \url{http://www.cips-cl.org/} 上发布额外的说明。请定期检查该网站。

手稿必须采用单栏格式。{\bf 请使用单倍行距。} 所有页面应直接从顶部边距开始。有关第一页格式的要求,请参见后面的指南。手稿的长度不得超过第~\ref{sec:length}节中描述的最大页数限制。
请勿对页面进行编号。
\subsection{电子资源}

我们强烈建议您使用 \LaTeX{} 并配合官方的 CCL 2024 样式文件 (ccl2024-en.sty) 和参考文献样式 (ccl.bst) 来准备您的 PDF 文件。这些文件可以在 ccl-2022-template.zip 中找到,下载地址为 \url{http://www.cips-cl.org/}。您还可以在 ccl2024.zip 中找到您当前阅读的文档 (ccl2024-en.pdf) 及其 \LaTeX{} 源代码 (ccl2024-en.tex)。
\subsection{电子稿件的格式}
\label{sect:pdf}

在制作电子稿件时,必须使用 Adobe 的可移植文档格式(PDF)。PDF 文件通常通过使用 \LaTeX{} 的 \textit{pdflatex} 命令生成。如果你的 \LaTeX{} 版本生成的是 Postscript 文件,可以使用 \textit{ps2pdf} 或 \textit{dvipdf} 将其转换为 PDF。在 Windows 系统中,也可以使用 Adobe Distiller 来生成 PDF。

请确保你的 PDF 文件中包含所有必要的字体(尤其是树状图、符号以及非拉丁字符所用的字体)。在打印或创建 PDF 文件时,通常在打印机设置中会有一个选项,可选择不包含、包含全部或仅包含非标准字体。请务必选择包含 \textbf{所有字体} 的选项。 \textbf{在发送之前,请在一台不同于生成该 PDF 文件的计算机上打印以测试该文件的可用性。} 此外,一些文字处理器可能会生成非常大的 PDF 文件,其中每一页都被渲染为图像。这类图像在复制时可能效果较差。在这种情况下,请尝试使用其他方式生成 PDF。在某些系统中,可以安装一个 Postscript 打印机驱动程序,将文档发送到打印机并指定“输出到文件”,然后将该文件转换为 PDF。

在排版论文时,\textbf{指定 A4 纸张格式}(21 cm × 29.7 cm)至关重要。例如,在使用 {\tt dvips} 时,应指定 {\tt -t a4}。

如果你无法满足上述电子投稿的要求,请尽快联系出版负责人。
```latex
\subsection{版面布局}
\label{ssec:layout}

按单栏格式排版,页面格式与本说明书的排版一致。A4纸的页面尺寸如下:

\begin{itemize}
\item 左右边距:2.5 厘米
\item 上边距:2.5 厘米
\item 下边距:2.5 厘米
\item 宽度:16.0 厘米
\item 高度:24.7 厘米
\end{itemize}

\noindent 请勿提交其他纸张尺寸的稿件。如果您无法满足上述电子提交的要求,请尽快联系上方的出版共同主席。
```
\subsection{字体}

为了保持统一性,应使用 Adobe 的 {\bf Times Roman} 字体。在 \LaTeX2e{} 中,通过在导言区加入

\begin{quote}
\begin{verbatim}
\usepackage{times}
\usepackage{latexsym}
\end{verbatim}
\end{quote}
来实现。如果 Times Roman 不可用,则使用 {\bf Computer Modern Roman}(\LaTeX2e{} 的默认字体)。请注意,后者的密度大约比 Adobe 的 Times Roman 字体低 10\%。 

\begin{table}[h]
\begin{center}
\begin{tabular}{|l|rl|}
\hline \bf Type of Text & \bf Font Size & \bf Style \\ \hline
paper title & 15 pt & bold \\
author names & 12 pt & bold \\
author affiliation & 12 pt & \\
the word ``Abstract'' & 12 pt & bold \\
section titles & 12 pt & bold \\
document text & 11 pt  &\\
captions & 11 pt & \\
sub-captions & 9 pt & \\
abstract text & 11 pt & \\
bibliography & 10 pt & \\
footnotes & 9 pt & \\
\hline
\end{tabular}
\end{center}
\caption{\label{font-table} 字体指南。}
\end{table}
\subsection{第一页}
\label{ssec:first}

将标题、作者姓名和单位信息居中排布于页面上。

不要使用脚注来注明单位信息。不要包含提交过程中分配的论文编号。审稿版本中不得包含作者姓名或单位信息。

{\bf 标题}:将标题居中放置在第一页顶部,使用 15 pt 加粗字体。(有关字体大小和样式的完整指南,请参见 Table~\ref{font-table}。)较长的标题应分成两行输入,中间不要留空行。大致将标题放置在距页面顶部约 2.5 cm 的位置,接着是一行空行,然后是作者姓名,再下一行为单位信息。不要仅使用名字的首字母(允许使用中间名首字母)。不要将姓氏全部大写(例如,应使用 ``Schlangen'' 而非 ``SCHLANGEN'')。同样,除约定俗成需大写的专有名称(如 ``BLEU'')外,不要将标题和章节标题全部大写。单位信息应包含作者的完整地址,并尽可能包含电子邮箱地址。正文应从距页面顶部 7.5 cm 处开始。

标题、作者姓名和地址应与在电子论文提交网站中填写的信息完全一致,以保持会议所有出版物中的作者信息一致性。如果存在不一致,出版联合主席可能会在未经您同意的情况下进行修改;因此,请务必仔细核对信息是否一致。

{\bf 摘要}:在单位信息与正文之间输入摘要内容。摘要的文本宽度应比正文左右各小约 0.6 cm。将 {\bf Abstract} 一词置于摘要正文之上,并居中,使用 12 pt 加粗字体。摘要应简明扼要地总结论文的总体观点与结论,长度不应超过 200 个词。摘要文本应使用 11 pt 字体。

{\bf 正文}:在摘要之后立即开始输入正文内容,按照本文件所示的单栏格式排版。不要包含页码。

{\bf 缩进}:新段落开头请使用缩进。正文和小节标题使用 11 pt 字体,章节标题使用 12 pt 字体,标题使用 15 pt 字体。

{\bf 许可声明}:在最终定稿版本的第一页中,以未编号的脚注形式包含许可声明。详细内容及说明请参见 Section~\ref{licence}。
\subsection{章节}

{\bf 标题}:请按照本文件所示的格式输入并标注章节和小节标题。使用编号章节(阿拉伯数字),以便于交叉引用。小节使用“章节号.小节号”的格式编号,均为阿拉伯数字。不要对小小节进行编号。

{\bf 引文}:正文中的引用应以括号形式出现,如~\cite{Gusfield:97},或者如果作者姓名已在正文中出现,则使用 Gusfield~\shortcite{Gusfield:97}。在出现歧义的情况下,请在年份后附加小写字母。双作者应如~\cite{Aho:72} 所示处理,但当作者多于两人时,应使用~\cite{Chandra:81} 的格式。多个引用应合并,如~\cite{Gusfield:97,Aho:72}。同时请避免将完整的文献引用用作句子成分。我们建议您使用
\begin{quote}
``Gusfield \shortcite{Gusfield:97} 表明...''
\end{quote}
来替代
\begin{quote}
  ``\cite{Gusfield:97} 显示了 ...''
\end{quote}

如果您使用提供的 \LaTeX{} 和 Bib\TeX{} 样式文件,可以使用命令 \verb|\newcite| 来生成 “作者(年份)” 形式的引用。

由于评审过程采用双盲评审,提交版本的论文中不得包含作者姓名和机构信息。此外,诸如
\begin{quote}
``我们之前展示了 \cite{Gusfield:97} ...''
\end{quote}
这类会暴露作者身份的自引用应当避免。请改用类似
\begin{quote}
``Gusfield \shortcite{Gusfield:97} 先前曾表明 ...''
\end{quote}
的引用方式。

\textbf{请勿使用匿名引用},并且在提交审稿版本时请不要包含以下内容:致谢、项目名称、资助编号、近三周内才公开或即将公开、可能影响匿名性的资源或工具的名称或网址。
不符合上述要求的论文可能会被拒绝评审。但这些内容可以在最终提交的 camera-ready 版本中包含。

\textbf{参考文献}:请将完整的参考文献汇总在标题为 {\bf References} 的部分;将该部分置于附录之前(除非附录中包含引用文献)。参考文献应按第一作者姓氏的字母顺序排列,而非按照在正文中出现的顺序排列。请尽可能提供完整且统一格式的引文,例如 {\em Computational Linguistics\/} 或 {\em Publication Manual of the American Psychological Association\/}~\cite{APA:83} 中所使用的格式。建议使用作者的全名而非姓名缩写。常见计算机科学期刊的缩写可参考 ACM 的 {\em Computing Reviews\/}~\cite{ACM:83}。

提供的 \LaTeX{} 和 Bib\TeX{} 样式文件大致符合美国心理学会的格式,支持如上所述的常规引用、简略引用和多项引用。

{\bf 附录}:如有附录,请将其置于正文和参考文献之后(见上文说明)。附录按字母顺序编号,并提供具有信息性的标题,例如:{\bf 附录 A. 附录标题}。
\subsection{脚注}

{\bf 脚注}:脚注应置于页面底部,并使用 9 pt 字体。它们可以使用数字编号,也可以用星号或其他符号表示。\footnote{This is how a footnote should appear.} 脚注应与正文之间用一条分隔线隔开。\footnote{Note the line separating the footnotes from the text.}
\subsection{图形}

{\bf 插图}:如果可能的话,将图表、表格和照片放置在论文中首次讨论的位置,而不是放在文末。
不建议使用彩色插图,除非您已经验证它们在黑白打印时仍然能够清晰可辨。

{\bf 图注}:为每个插图提供图注;按顺序编号,每个图注的形式为:“图 1. 图的图注。” “表 1. 表的图注。” 将图表的图注放置在正文下面,使用 11 pt 字体。

狭窄的图形与单栏格式结合使用可能会导致较大的空白区域, 
例如,参见表~\ref{font-table} 两侧的宽边距。
如果您有多个内容相关的图形,最好将它们合并为一个图形。
您可以在子图下方使用子图注来标识每个子图,编号为 (a)、(b)、(c) 等,并使用 9 pt 字体。
\LaTeX{} 包括 wrapfig、subfig、subtable 和/或 subcaption 可能会非常有用。
\subsection{许可声明}
\label{licence}

我们要求作者将其
定稿论文授权
在
创意共享署名 4.0 国际许可证
(CC-BY)下发布。
这意味着作者(版权所有者)保留版权,但
授予每个人
适应和再分发其论文的权利,
前提是作者需被注明,并列出修改内容。
换句话说,这个许可让研究人员可以在没有法律问题的情况下使用研究论文进行研究。
有关许可条款,请参阅
\url{http://creativecommons.org/licenses/by/4.0/}。

\begin{itemize}
    \item 本作品采用创作共用 4.0 国际许可协议。许可详情请见:\url{http://creativecommons.org/licenses/by/4.0/}。

\end{itemize}

我们强烈建议您将您的论文按照上述 CC 许可进行授权。
(请注意,此许可声明仅适用于已接受论文的最终版本,
并不要求对提交审稿的论文进行授权。)
\section{投稿长度}
\label{sec:length}

投稿的最大长度为 10 页(A4 纸张),参考文献部分页数不限。被接收论文的作者将在最终定稿版本中获得额外的页面空间,以反映因评审意见而需做出的修改所需的空间。

不符合指定长度和格式要求的论文可能会被直接拒稿,且不进入评审流程。
\section*{致谢}

致谢部分应紧接在参考文献之前。不要为致谢部分编号。提交论文审稿时不包括此部分。

\begin{thebibliography}{}

\bibitem[\protect\citename{Aho and Ullman}1972]{Aho:72}
Alfred~V. Aho and Jeffrey~D. Ullman.
\newblock 1972.
\newblock {\em The Theory of Parsing, Translation and Compiling}, volume~1.
\newblock Prentice-{Hall}, Englewood Cliffs, NJ.

\bibitem[\protect\citename{{American Psychological Association}}1983]{APA:83}
{American Psychological Association}.
\newblock 1983.
\newblock {\em Publications Manual}.
\newblock American Psychological Association, Washington, DC.

\bibitem[\protect\citename{{Association for Computing Machinery}}1983]{ACM:83}
{Association for Computing Machinery}.
\newblock 1983.
\newblock {\em Computing Reviews}, 24(11):503--512.

\bibitem[\protect\citename{Chandra \bgroup et al.\egroup }1981]{Chandra:81}
Ashok~K. Chandra, Dexter~C. Kozen, and Larry~J. Stockmeyer.
\newblock 1981.
\newblock Alternation.
\newblock {\em Journal of the Association for Computing Machinery},
  28(1):114--133.

\bibitem[\protect\citename{Gusfield}1997]{Gusfield:97}
Dan Gusfield.
\newblock 1997.
\newblock {\em Algorithms on Strings, Trees and Sequences}.
\newblock Cambridge University Press, Cambridge, UK.

\end{thebibliography}

\end{CJK*}
\end{document}
