\begin{abstract}

  对于大型语言模型(LLM)的对齐,传统方法通常依赖于昂贵的训练和人工偏好注释。自我对齐旨在通过使模型自行对齐来减少这些开销。为了进一步降低成本,并在没有昂贵调优或注释的情况下实现对齐,我们提出了一种新的无需调优的自我对齐方法——动态奖励与提示优化(\ours)。我们的方法利用基于搜索的优化框架,使LLM能够反复自我改进,并制定最佳的对齐指令,且无需额外的训练或人工干预。\ours的核心是一个动态奖励机制,它识别并纠正模型特定的对齐弱点,从而使LLM能够高效地适应不同的对齐挑战。在对八个最近发布的LLM(包括开源和闭源模型)进行的实证评估中,\ours显著提高了对齐性能,且基础模型的表现超过了经过SFT/RLHF调优的对应模型。此外,通过\ours自动优化的提示超过了人工专家精心设计的提示,进一步验证了我们方法的有效性。我们的研究结果突显了当前LLM通过推理时优化实现自适应自我对齐的巨大潜力,补充了基于调优的对齐方法。\footnote{代码可用:\url{https://github.com/Singla17/DRPO}}

\end{abstract}
