\section{预备知识}

\subsection{定义}

\cpar{原生多模态模型 (NMMs):}
从头开始同时训练所有模态的模型,而不依赖于预训练的LLMs或视觉编码器。我们的重点是代表性的图像和文本模态,其中模型处理文本和图像作为输入,并生成文本作为输出。

\cpar{早期融合:} 从一开始就启用多模态交互,几乎不使用任何特定于模态的参数(例如,除了一个用于图像分块的线性层)。使用单一的变压器模型,这种方法处理原始的多模态 \edit{输入——标记化文本和连续的图像块——而不进行图像离散化。} \edit{我们} 将主要的变压器称为解码器。

\cpar{后期融合:} 将多模态交互 \edit{推迟到} 更深层次,通常是在单独的单模态组件 \edit{处理} 每个模态后独立进行(例如,连接到LLM的视觉编码器)。

\cpar{与模态无关的路由:} 在稀疏专家混合模型中,\edit{与模态无关的}路由指的是依赖于一个与模型联合训练的学习路由模块。

\cpar{与模态相关的路由:} 基于预定义规则的路由,例如基于模态类型的路由(例如,视觉标记、标记标记)。

\input{tables/notations}
\subsection{尺度规律}
我们旨在理解NMM的尺度属性以及不同架构选择如何影响权衡。为此,我们在~\citet{kaplan2020scaling, hoffmann2022training}提出的尺度规律框架内分析我们的模型。我们根据总参数数量计算FLOPs,使用了先前工作中采用的近似公式 \(C = 6ND\)~\citep{hoffmann2022training,abnar2025parameters}。然而,我们对这一估算进行了修改,以适应我们的设置:对于后融合模型,FLOPs的计算公式为 \(6(N_vD_v + ND)\)。

我们考虑一种设置,其中,在给定计算预算 \(C\) 的情况下,我们的目标是预测模型的最终\edit{损失},并确定最佳的参数数量\edit{和}训练令牌数量。与先前关于LLM尺度研究的一致~\citep{hoffmann2022training},我们假设最终模型损失与模型大小(\(N\))和训练令牌(\(D\))之间存在幂律关系:

\begin{equation}
\label{eq:scaling_laws}
    L = E + \frac{A}{N^{\alpha}} + \frac{B}{D^{\beta}}.
\end{equation}

\noindent 其中,\(E\)表示数据集上可实现的最低损失,而 \(\frac{A}{N^{\alpha}}\) 捕捉了增加参数数量的效果,其中更大的模型导致较低的损失,改进的速率由 \(\alpha\) 控制。类似地,\(\frac{B}{D^{\beta}}\) 说明了更多令牌的好处,\(\beta\) 决定了改进的速率。此外,我们假设计算预算(FLOPs)与 \(N\) 和 \(D\) 之间存在线性关系(\(C \propto ND\))。这进一步导致了在\cref{tab:power_laws}中详细描述的幂律关系。


\begin{figure}[t!]
    \centering
    \captionsetup{type=figure}
    \begin{subfigure}[t]{0.48\linewidth}
        \centering
        \includegraphics[width=1.02\linewidth]{assets/early/3d_scaling_early.pdf}
    \end{subfigure}
    \hfil
    \begin{subfigure}[t]{0.48\linewidth}
        \centering
        \includegraphics[width=1.02\linewidth]{assets/early/3d_scaling_late.pdf}
    \end{subfigure}
    \vspace{5pt}
    \setlength{\fboxsep}{0.5pt}
    \setlength{\fboxrule}{0pt}
    \caption{\textbf{适用于\fbox{\colorbox{CustomC_Light1!20}{\strut
    提前融合}}和\fbox{\colorbox{CustomD_Light1!20}{延迟融合\strut}}
    原生多模态模型的缩放规律。} 每个点代表一个模型(300M到3B参数),在不同数量的词元(250M到400B)上进行训练。我们在\edit{交错数据(Obelics)、图像-标题数据(HQITP)以及仅文本数据(DCLM)}的有效验证集上报告平均交叉熵损失。}
    \label{fig:early_vs_late_scaleflops_3d}
\end{figure}


\begin{table}[h]
    \centering
    \setlength{\tabcolsep}{16pt}
    \renewcommand{\arraystretch}{1}
    \resizebox{1\linewidth}{!}{
    \begin{tabular}{lccrc}
        Data type & dataset & \#samples & sampling prob. \\
         \shline
         \multirow{3}{*}{Image-Caption} &  DFN~\citep{fang2023data} & 2B & 27\%
         \\
          & COYO~{\citep{kakaobrain2022coyo700m}} &  600M & 11.25\% \\
          & HQITP  & 400M & 6.75\% \\
          Interleaved & Obelics \citep{laurenccon2024obelics}  & 141M Docs &
          45\% \\
          Text & DCLM \citep{li2024datacomp} & 6.6T Toks & 10\% \\

    \end{tabular}} \caption{\textbf{预训练数据混合。}除非另有说明,训练数据混合中包含 45\%、45\% 和 10\% 的图像标题、交错文档和纯文本数据。}
    \label{tab:pretraining_datasets}
    \vspace{-5pt}
\end{table}
\subsection{实验设置}
\edit{我们的模型} 基于自回归变换器架构~\citep{vaswani2017attention},使用SwiGLU FFN~\citep{shazeer2020glu} 和QK-Norm~\citep{dehghani2023scaling},遵循~\citet{li2024datacomp}。在早期融合模型中,图像块被线性映射以匹配文本标记维度,而后期融合遵循CLIP架构~\citep{radford2021learning}。我们对文本标记采用因果注意力,对图像标记采用双向注意力,发现这样效果更好。训练在公开和私人多模态数据集的混合上进行,包括DCLM \citep{li2024datacomp}、Obelics \citep{laurenccon2024obelics}、DFN \citep{fang2023data}、COYO \citep{kakaobrain2022coyo700m},以及一个私有的高质量图像-文本对(HQITP)集合(见\cref{tab:pretraining_datasets})。图像被\edit{调整大小}至224×224分辨率,图像块大小为14×14。我们使用1k的上下文长度来处理多模态序列。为了提高训练效率,我们使用\texttt{bfloat16}、完全分片数据并行(FSDP)\citep{zhao2023pytorch}、\edit{激活}检查点和梯度累积。我们还使用序列打包来处理图像标注数据集,以减少填充标记的数量。与之前的工作~\citep{hoffmann2022training,aghajanyan2023scalingmm,abnar2025parameters}类似,我们在交替的(Obelics)、图像-标注(HQITP)和仅文本数据(DCLM)上评估性能。更多的实现细节请参见~\cref{app:implementation_details}。
