\begin{figure}[t!]
    \centering
    \captionsetup{type=figure}
    \begin{subfigure}[t]{0.48\linewidth}
        \input{graphs/early_late/loss_vs_flops}
    \end{subfigure}
    \hfill
    \begin{subfigure}[t]{0.48\linewidth}
        \input{graphs/early_late/d_n_ratio_vs_flops}
    \end{subfigure}
    \vspace{-3mm}
    \caption{\textbf{原生多模态模型的规模特性。} 基于 \cref{sec:scaling_laws_early} 中的规模法则研究,我们观察到:(1)早期和后期融合模型在使用相同计算预算 $C$(以 FLOPs 为单位)进行训练时,提供相当的验证损失 $L$;(2)这一性能是通过在参数 $N$ 和训练标记数量 $D$ 之间的不同权衡实现的,其中早期融合模型需要更少的参数。 \edit{;(3)稀疏的早期融合模型在给定 FLOP 预算的情况下,能够实现更低的损失,并且需要更多的训练标记。} }
    \label{fig:teaser}
\end{figure}
