\begin{figure}[t!]
    \centering
    \captionsetup{type=figure}
    \begin{subfigure}[t]{0.32\linewidth}
        \input{graphs/early_late_init/early_vs_late_init_scaledata_getty}
    \end{subfigure}
    \begin{subfigure}[t]{0.32\linewidth}
        \input{graphs/early_late_init/early_vs_late_init_scaledata_obelics}
    \end{subfigure}
    \begin{subfigure}[t]{0.32\linewidth}
        \input{graphs/early_late_init/early_vs_late_init_scaledata_dclm}
    \end{subfigure}

    \makebox[0.9\linewidth]{
        \begin{tikzpicture}
            \begin{axis}[
                hide axis,
                xmin=0, xmax=0.5, ymin=0, ymax=1,
                legend columns=2,
                legend style={
                    at={(0.5, 1)},
                    anchor=north,
                    /tikz/every even column/.append style={column sep=0.2cm},
                    scale=0.5,
                    cells={align=left}, font=\footnotesize,
                },
            ]
            \addlegendimage{LateGradStart!75!LateGradEnd, thick, solid, mark=*, mark size=1.5pt}
            \addlegendentry{Late-init}

\addlegendimage{legend early_2_2b style}
            \addlegendentry{Early-Init}
            \end{axis}
        \end{tikzpicture}
    }

    \vspace{-5cm}
    \caption{\textbf{编码器和解码器初始化时的早期融合与后期融合比较。} 当训练时间更长时,早期融合可以达到后期融合模型的性能。然而,在图像字幕数据上差距更大。}
    \label{fig:early_vs_late_init_scaledata}
\end{figure}
