\begin{figure*}[t!]
    \centering
    \captionsetup{type=figure}
    \begin{subfigure}[t]{0.33\linewidth}
        \input{graphs/early_late/early_vs_late_scaleflops_getty}
    \end{subfigure}
    \begin{subfigure}[t]{0.32\linewidth}
        \input{graphs/early_late/early_vs_late_scaleflops_obelics}
    \end{subfigure}
    \begin{subfigure}[t]{0.32\linewidth}
    \input{graphs/early_late/early_vs_late_scaleflops_dclm}
    \end{subfigure}
    \begin{center}
        \ref{sharedlegend}
    \end{center}
    \caption{\textbf{早期融合与晚期融合:扩展训练FLOP数。} 我们比较了在扩展模型参数数量和训练令牌数量时,早期融合和晚期融合模型的表现。总体而言,早期融合在较小的模型规模下表现出轻微的优势,并且随着参数数量 $N$ 的增加,差距逐渐缩小。}
    \label{fig:early_vs_late_scaleflops}
    \vspace{3mm}
\end{figure*}
