\newcommand{\sampleheight}{42pt}
\newcommand{\covheight}{46pt}
\newcommand{\marginalheight}{50pt}
\setlength{\tabcolsep}{4pt}
\begin{figure}[t]
  \centering
  \hspace{-1.2em}
  \scalebox{0.9}{
  \begin{centering}
    \begin{tabular}{p{51pt}
      @{\hspace{10pt}}m{2pt}l
      @{\hspace{5pt}}l
      @{\hspace{10pt}}m{2pt}l
      @{\hspace{10pt}}m{2pt}l}
        \raisebox{18pt}{\small$\texttt{Ising}$} &
        \raisebox{34pt}{\rotatebox{90}{\tiny input value}} &
        \includegraphics[height=\sampleheight]{figures/task/samples_long/ising.pdf} &
        \includegraphics[height=\sampleheight]{figures/task/samples_short/ising.pdf} &
        \raisebox{38pt}{\rotatebox{90}{\tiny input dimension}} &
        \includegraphics[height=\covheight]{figures/task/cov/ising.pdf} &
        \raisebox{40pt}{\rotatebox{90}{\tiny $p(X_i)$}} &
        \raisebox{-4pt}{\includegraphics[height=\marginalheight]{figures/task/marginal/ising.pdf}} \\
        \noalign{\vskip -36pt}
        \raisebox{18pt}{\small$\texttt{NLGP}(0.01)$} &
        \raisebox{34pt}{\rotatebox{90}{\tiny input value}} &
        \includegraphics[height=\sampleheight]{figures/task/samples_long/gaussian.pdf} &
        \includegraphics[height=\sampleheight]{figures/task/samples_short/gaussian.pdf} &
        \raisebox{38pt}{\rotatebox{90}{\tiny input dimension}} &
        \includegraphics[height=\covheight]{figures/task/cov/gaussian.pdf} &
        \raisebox{40pt}{\rotatebox{90}{\tiny $p(X_i)$}} &
        \raisebox{-4pt}{\includegraphics[height=\marginalheight]{figures/task/marginal/gaussian.pdf}} \\
        \noalign{\vskip -36pt}
        \raisebox{18pt}{\small $\texttt{Kur}(5)$} &
        \raisebox{34pt}{\rotatebox{90}{\tiny input value}} &
        \includegraphics[height=\sampleheight]{figures/task/samples_long/alg5.pdf} &
        \includegraphics[height=\sampleheight]{figures/task/samples_short/alg5.pdf} &
        \raisebox{38pt}{\rotatebox{90}{\tiny input dimension}} &
        \includegraphics[height=\covheight]{figures/task/cov/alg5.pdf} &
        \raisebox{40pt}{\rotatebox{90}{\tiny $p(X_i)$}} &
        \raisebox{-4pt}{\includegraphics[height=\marginalheight]{figures/task/marginal/alg5.pdf}} \\
        \noalign{\vskip -37pt}
        &&
        \hspace{25pt}\tiny input dimension &
        \hspace{25pt}\tiny input dimension & &
        \hspace{3pt}\tiny input dimension & &
        \hspace{37pt}\tiny input value \\
  \end{tabular}
  \end{centering}
  }
  \caption{
    从左到右:
    长尺度和短尺度样本 $\mathbf{x}$,
    单一尺度的协方差 $\Sigma$,
    以及数据模型的边缘分布 $p(X_i)$,如 \cref{sec:task} 所述:
    Ising 模型(左、右样本分别为 $J=1.2, 0.3$),
    非线性高斯过程~\parencite[NLGP;~][]{ingrosso2022data},
    以及可控峰度模型 \texttt{Kur}
    (左、右样本分别为 $\xi=5, 1$)。
    \emph{
    每个模型生成的样本以零为中心,且其协方差可以被约束为相似,
    但具有不同的高阶统计量,从维度方向的边缘分布中可以看出这一点。
    }
}
  \label{fig:task}
  \vspace{-10pt}
\end{figure}
